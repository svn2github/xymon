\chapter{HOBBIT-ENADIS.CGI}

section{NAME}
 hobbit-enadis.cgi - CGI program to enable/disable Hobbit tests �\section{SYNOPSIS}
\textbf{hobbit-enadis.cgi (invoked via CGI from webserver)}


 �
\section{DESCRIPTION}
\textbf{hobbit-enadis.cgi} is a CGI tool for disabling and enabling
hosts and tests monitored by Hobbit. You can disable monitoring of a
single test, all tests for a host, or multiple hosts - immediately or
at a future point in time. 


  hobbit-enadis.cgi runs as a CGI program, invoked by your
  webserver. It is normally run via a wrapper shell-script in the
  secured CGI directory for Hobbit. 



  hobbit-enadis.cgi is the back-end script for the enable/disable form
  present on the ``info'' status-pages. It can also run in
  ``stand-alone'' mode, in which case it displays a web form allowing
  users to select what to enable or disable. 



 


 �
\section{OPTIONS}
\begin{description}
\item[--no-cookies] Normally, hobbit-enadis.cgi uses a cookie sent by
  the browser to initially filter the list of hosts presented. If this
  is not desired, you can turn off this behaviour by calling
  bb-ack.cgi with the --no-cookies option. This would normally be
  placed in the CGI\_ENADIS\_OPTS setting in \emph{hobbitcgi.cfg(5)}



 

\item[--env=FILENAME] Load the environment from FILENAME before
  executing the CGI. 


 

\item[--area=NAME] Load environment variables for a specific area. NB:
  if used, this option must appear before any --env=FILENAME option. 

\end{description}
�\

subsection{FILES}
\begin{description}
\item [\$BBHOME/web/maint\_\\{header,form,footer\\}] HTML template header 

 


\end{description}
�\section{BUGS}
 When using alternate pagesets, hosts will only show up on the
 Enable/Disable page if this is accessed from the primary page in
 which they are defined. So if you have hosts on multiple pages, they
 will only be visible for disabling from their main page which is not
 what you would expect. 


 �
\section{SEE ALSO}
hobbit(7) 

 
