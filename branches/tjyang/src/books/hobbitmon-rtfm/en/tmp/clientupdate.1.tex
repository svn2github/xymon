



\chapter{CLIENTUPDATE}

\section{CLIENTUPDATE}
 Section: User Commands (1) 
Updated: Version Exp: 11 Jan 2008 
Index Return to Main Contents 
�\section{NAME}
 clientupdate - Hobbit client update utility �\section{SYNOPSIS}
\textbf{clientupdate [options]}


 �
\section{DESCRIPTION}
\textbf{clientupdate}
 is part of the Hobbit client. It is responsible for updating an existing client installation from a central repository of client packages stored on the Hobbit server. 

  When the Hobbit client sends a normal client report to the Hobbit server, the server responds with the section of the \emph{client-local.cfg(5)}
 file that is relevant to this client. Included in this may be a ``clientversion'' value. The clientversion received from the server is compared against the current clientversion installed on the client, as determined by the contents of the \$BBHOME/etc/clientversion.cfg file. If the two versions are not identical, clientupdate is launched to update the client installation. 


 �
\section{OPTIONS}
\begin{description}
\item[--level] Report the current clientversion. 

 

\item[--update=NEWVERSION] Attempt to update the client to NEWVERSION by fetching this version of the client software from the Hobbit server. 

 

\item[--reexec] Used internally during the update process, see \textbf{OPERATION}
 below. 

 

\item[--remove-self] Used internally during the update process. This option causes the running clientupdate utility to delete itself - it is used during the update to purge a temporary copy of the clientupdate utility that is installed in \$BBTMP. 

 


\end{description}
�\section{USING CLIENTUPDATE IN HOBBIT}
 To manage updating clients without having to logon to each server, you can use the clientupdate utility. This is how you setup the release of a new client version. 

 \begin{description}
\item[Create the new client] Setup the new client \$BBHOME directory, e.g. by copying an existing client installation to an empty directory and modifying it for your needs. It is a good idea to delete all files in the tmp/ and logs/ directories, since there is no need to copy these over to all of the clients. Pay attention to the etc/ files, and make sure that they are suitable for the systems where you want to deploy this new client. You can add files - e.g. extension scripts in the ext/ directory - but the clientupdate utility cannot delete or rename files. 

 

\item[Package the client] When your new client software is ready, create a tar-file of the new client. All files in the tar archive must have filenames relative to the clients' \$BBHOME (usually, ~hobbit/client/). Save the tarfile on the Hobbit server in ~hobbit/server/download/somefile.tar. Dont compress it. It is recommended that you use some sort of operating-system and version-numbering scheme for the filename, but you can choose whatever filename suits you - the only requirement is that it must end with ``.tar''. The part of the filename preceding ``.tar'' is what Hobbit will use as the ``clientversion'' ID. 

 

\item[Configure which hosts receive the new client] In the \emph{client-local.cfg(5)}
 file, you must now setup a \textbf{clientversion:ID}
 line where the \textbf{ID}
 matches the filename you used for the tar-file. So if you have packaged the new client into the file \textbf{linux.v2.tar}
, then the corresponding entry in client-local.cfg would be \textbf{clientversion:linux.v2}
. 

 

\item[Wait for hobbitd to reload client-local.cfg] hobbitd will automatically reload the client-local.cfg file after at most 10 minutes. If you want to force an immediate reload, send a SIGHUP signal to the hobbitd process. 

 

\item[Wait for the client to update] The next time the client contacts the Hobbit server to send the client data, it will notice the new clientversion setting in client-local.cfg, and will run \textbf{clientupdate}
 to install the new client software. So when the client runs the next time, it will use the new client software. 

 


\end{description}
�
\section{OPERATION}
\textbf{clientupdate}
 runs in two steps: 

 \begin{description}
\item[Re-exec step] The first step is when clientupdate is first invoked from the hobbitclient.sh script with the ``--re-exec'' option. This step copies the clientupdate program from \$BBHOME/bin/ to a temporary file in the \$BBTMP directory. This is to avoid conflicts when the update procedure installs a new version of the clientupdate utility itself. Upon completion of this step, the clientupdate utility automatically launches the next step by running the program from the file in \$BBTMP. 

 

\item[Update step] The second step downloads the new client software from the Hobbit server. The new software must be packed into a tar file, which clientupdate then unpacks into the \$BBHOME directory. 

 


\end{description}
�
\section{ENVIRONMENT VARIABLES}
 clientupdate uses several of the standard Hobbit environment variables, including \textbf{BBHOME}
 and \textbf{BBTMP}
. 

 �
\section{SEE ALSO}
hobbit(7), bb(1), client-local.cfg(5) 

 


  
�
\section{Index}
\begin{description}
\item[NAME]
\item[SYNOPSIS]
\item[DESCRIPTION]
\item[OPTIONS]
\item[USING CLIENTUPDATE IN HOBBIT]
\item[OPERATION]
\item[ENVIRONMENT VARIABLES]
\item[SEE ALSO]

\end{description}
 
 This document was created by man2html, using the manual pages. 
 Time: 16:21:46 GMT, January 11, 2008 

