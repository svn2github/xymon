



\chapter{HOBBITD\_CHANNEL}

\section{HOBBITD\_CHANNEL}
 Section: Maintenance Commands (8) 
Updated: Version Exp: 11 Jan 2008 
Index Return to Main Contents 
�\section{NAME}
 hobbitd\_channel - Feed a hobbitd channel to a worker module �\section{SYNOPSIS}
\textbf{hobbitd\_channel --channel=CHANNEL [options] workerprogram [worker-options]}


 �
\section{DESCRIPTION}
 hobbitd\_channel hooks into one of the \emph{hobbitd(8)}
 channels that provide information about events occurring in the Hobbit system. It retrieves messages from the hobbitd daemon, and passes them on to the \textbf{workerprogram}
 on the STDIN (file descripter 1) of the worker program. Worker programs can then handle messages as they like. 

  A number of worker programs are shipped with hobbitd, e.g. \emph{hobbitd\_filestore(8)}
 \emph{hobbitd\_history(8)}
 \emph{hobbitd\_alert(8)}
 \emph{hobbitd\_rrd(8)}
 \emph{hobbitd\_client(8)}



  If you want to write your own worker module, a sample worker module is provided as part of the hobbitd distribution in the hobbitd\_sample.c file. This illustrates how to easily fetch and parse messages. 


 �
\section{OPTIONS}
 hobbitd\_channel accepts a few options. 

 \begin{description}
\item[--channel=CHANNELNAME] Specifies the channel to receive messages from, only one channel can be used. This option is required. The following channels are available:  
 ``status'' receives all Hobbit status- and summary-messages  
 ``stachg'' receives information about status changes  
 ``page'' receives information about statuses triggering alerts  
 ``data'' receives all Hobbit ``data'' messages  
 ``notes'' receives all Hobbit ``notes'' messages  
 ``enadis'' receives information about hosts being disabled or enabled.  
 ``client'' receives all data sent from Hobbit client systems 

 

\item[--net=PEERSERVER:PEERPORT] Instead of launching a worker module as a local task, the messages in the channel are forwarded over a TCP/IP connection to another host where the worker module is running. This is typically used for sharing the load of the heavier worker modules across multiple systems, e.g. the hobbitd\_client and hobbitd\_rrd workers may receive data this way. With this option, the \textbf{workerprogram}
 parameter is ignored and may be omitted.  
 On the remote server, the worker modules are usually launched via inetd. 

 

\item[--daemon] hobbitd\_channel is normally started by \emph{hobbitlaunch(8)}
 as a task defined in the \emph{hobbitlaunch.cfg(5)}
 file. If you are not using hobbitlaunch, then starting hobbitd\_channel with this option causes it to run as a stand-alone background task. 

 

\item[--pidfile=FILENAME] If running as a stand-alone daemon, hobbitd\_channel will save the proces-ID of the daemon in FILENAME. This is useful for automated startup- and shutdown- scripts. 

 

\item[--env=FILENAME] Loads the environment variables defined in FILENAME before starting hobbitd\_channel. This is normally used only when running as a stand-alone daemon; if hobbitd\_channel is started by hobbitlaunch, then the environment is controlled by the task definition in the \emph{hobbitlaunch.cfg(5)}
 file. 

 

\item[--log=FILENAME] Redirect output to this log-file. 

 

\item[--debug] Enable debugging output. 

 


\end{description}
�
\section{FILES}
 This program does not use any configuration files. 

 �
\section{SEE ALSO}
hobbitd(8), hobbit(7) 

 


  
�
\section{Index}
\begin{description}
\item[NAME]
\item[SYNOPSIS]
\item[DESCRIPTION]
\item[OPTIONS]
\item[FILES]
\item[SEE ALSO]

\end{description}
 
 This document was created by man2html, using the manual pages. 
 Time: 16:21:46 GMT, January 11, 2008 

