



\chapter{HOBBITD\_CAPTURE}

\section{HOBBITD\_CAPTURE}
 Section: Maintenance Commands (8) 
Updated: Version Exp: 11 Jan 2008 
Index Return to Main Contents 
�\section{NAME}
 hobbitd\_capture - catch selected messages from a hobbitd channel �\section{SYNOPSIS}
\textbf{hobbitd\_channel --channel=status hobbitd\_capture [options]}


 �
\section{DESCRIPTION}
 hobbitd\_capture is a worker module for hobbitd, and as such it is normally run via the \emph{hobbitd\_channel(8)}
 program. It receives messages from hobbitd via stdin and filters them to select messages based on the hostname, testname or color of the status. By default the resulting messages are printed on stdout, but they can also be fed into a command for further processing. 

  hobbitd\_capture supports the \textbf{status}
, \textbf{data}
 and \textbf{hostdata}
 channels. 


 �
\section{OPTIONS}
\begin{description}
\item[--hosts=PATTERN] Select messages only from hosts matching PATTERN (regular expression). 

 

\item[--exhosts=PATTERN] Exclude messages from hosts matching PATTERN. If used with the --hosts option, then the hostname must match the --hosts pattern, but NOT the --exhosts pattern. 

 

\item[--tests=PATTERN] Select messages only from tests matching PATTERN (regular expression). 

 

\item[--extests=PATTERN] Exclude messages from tests matching PATTERN. If used with the --tests option, then the testname must match the --tests pattern, but NOT the --extests pattern. 

 

\item[--colors=COLOR[,color]] Select messages based on the color of the status message. Multiple colors can be listed, separated by comma. Default: Accept all colors. 

 

\item[--batch-command=COMMAND] Instead of printing the messages to stdout, feed them to COMMAND on stdin. COMMAND can be any command which accepts the mssage on standard input. 

 

\item[--batch-timeout=SECONDS] Collect messages until no messages have arrived in SECONDS seconds, before sending them to the --batch-command COMMAND. 

 

\item[--debug] Enable debugging output. 

 


\end{description}
�\section{SEE ALSO}
hobbitd\_channel(8), hobbitd(8), hobbit(7) 

 


  
�
\section{Index}
\begin{description}
\item[NAME]
\item[SYNOPSIS]
\item[DESCRIPTION]
\item[OPTIONS]
\item[SEE ALSO]

\end{description}
 
 This document was created by man2html, using the manual pages. 
 Time: 16:21:46 GMT, January 11, 2008 

