



\chapter{HOBBITSERVER.CFG}

\section{HOBBITSERVER.CFG}
 Section: File Formats (5) 
Updated: Version Exp: 11 Jan 2008 
Index Return to Main Contents 
�\section{NAME}
 hobbitserver.cfg - Hobbit environment variables 

 �
\section{DESCRIPTION}
 Hobbit programs use multiple environment variables beside the normal set of variables. The environment definitions are stored in the ~hobbit/server/etc/hobbitserver.cfg file. Each line in this file is of the form \textbf{NAME=VALUE}
 and defines one environment variable NAME with the value VALUE. 

 �
\section{ENVIRONMENT AREAS}
 In some cases it may be useful to have different values for an environment variable, depending on where it is used. This is possible by defining variables with an associated ``area''. Such definitions have the form \textbf{AREA/NAME=VALUE}
. 

  E.g. to define a special setup of the BBDISPLAY variable when it is used by an application in the ``management'' area, you would do this: \begin{description}
\item[]\begin{verbatim}

  BBDISP="127.0.0.1"            # Default definition
  management/BBDISP="10.1.0.5"  # Definition in the "management" area

\end{verbatim}


\end{description}



  Areas are invoked by using the ``--area'' option for all tools, or via the ENVAREA setting in the \emph{hobbitlaunch.cfg(5)}
 file. 


 �
\section{GENERAL SETTINGS}


 \begin{description}
\item[BBSERVERHOSTNAME] The fully-qualified hostname of the server that is running Hobbit. 

 

\item[BBSERVERWWWNAME] The hostname used to access this servers' web-pages, used to construct URL's in the Hobbit webpages. Default is the BBSERVERHOSTNAME. 

 

\item[BBSERVERIP] The public IP-address of the server that is running Hobbit. 

 

\item[BBSERVEROS] A name identifying the operating system of the Hobbit server. The known names are currently ``linux'', ``freebsd'', ``solaris'', ``hpux'', ``aix'' and ``osf''. 

 

\item[FQDN] If set to TRUE, Hobbit will use fully-qualified hostnames throughout. If set to FALSE, hostnames are stripped of their domain-part before being processed. It is \textbf{highly recommended}
 that you keep this set to TRUE. Default: TRUE. 

 

\item[BBLOGSTATUS] Controls how the HTML page for a status log is generated. If set to DYNAMIC, the HTML logs are generated on-demand by the \emph{bb-hostsvc.cgi(1)}
 script. If set to STATIC, you must activate the \emph{hobbitd\_filestore(8)}
 module (through an entry in the \emph{hobbitlaunch.cfg(5)}
 file) to create and store the HTML logs whenever a status update is received. Setting ``BBLOGSTATUS=STATIC'' is \textbf{discouraged}
 since the I/O load on the Hobbit server will increase significantly. 

 

\item[PINGCOLUMN] Defines the name of the column for ``ping'' test status. The data from the ``ping'' test is used internally by Hobbit, so it must be defined here so all of the Hobbit tools know which column to watch for this data. The default setting is PINGCOLUMN=conn. 

 

\item[INFOCOLUMN] Defines the name of the column for the ``info'' pages. 

 

\item[TRENDSCOLUMN] Defines the name of the column for the RRD graph pages. 

 

\item[RRDHEIGHT] The default height (in pixels) of the RRD graph images. Default: 120 pixels. 

 

\item[RRDWIDTH] The default width (in pixels) of the RRD graph images. Default: 576 pixels. 

 

\item[TRENDSECONDS] The graphs on the ``trends'' page show data for the past TRENDSECONDS seconds. Default: 172800 seconds, i.e. 48 hours. 

 

\item[HTMLCONTENTTYPE] The Content-type reported by the CGI scripts that generate web pages. By default, this it ``text/html''. If you have on-line help texts in character sets other than the ISO-8859-1 (western european) character set, it may be necessary to modify this to include a character set. E.g.  
 
���HTMLCONTENTTYPE=''text/html;�charset=euc-jp''  
 for a Japanese character sets. Note: Some webservers will automatically add this, if configured to do so. 

 

\item[HOLIDAYS] Defines the default set of holidays used if there is no ``holidays'' tag for a host in the bb-hosts file. Holiday sets are defined in the \emph{hobbit-holidays.cfg(5)}
 file. If not defined, only the default holidays (those defined outside a named holiday set) will be considered as holidays. 

 

\item[WEEKSTART] Defines which day is the first day of the week. Set to ``0'' for Sunday, ``1'' for Monday. Default: 1 (Monday). 

 


 


\end{description}
�
\section{DIRECTORIES}


 \begin{description}
\item[BBSERVERROOT] The top-level directory for the Hobbit installation. The default is the home-directory for the user running Hobbit. 

 

\item[BBSERVERLOGS] The directory for the Hobbit's own logfiles (NOT the status-logs from the monitored hosts). 

 

\item[BBHOME] The Hobbit server directory, where programs and configurations are kept. Default: \$BBSERVERROOT/server/ . 

 

\item[BBTMP] Directory used for temporary files. Default: \$BBHOME/tmp/ 

 

\item[BBWWW] Directory for Hobbit webfiles. The \$BBWEB URL must map to this directory. Default: \$BBHOME/www/ 

 

\item[BBNOTES] Directory for Hobbit notes-files. The \$BBNOTESSKIN URL must map to this directory. Default: \$BBHOME/www/notes/ 

 

\item[BBREP] Directory for Hobbit availability reports. The \$BBREPURL URL must map to this directory. Note also that your webserver must have write-access to this directory, if you want to use the \emph{bb-rep.cgi(1)}
 CGI script to generate reports on-demand. Default: \$BBHOME/www/rep/ 

 

\item[BBSNAP] Directory for Hobbit snapshots. The \$BBSNAPURL URL must map to this directory. Note also that your webserver must have write-access to this directory, if you want to use the \emph{bb-snapshot.cgi(1)}
 CGI script to generate snapshots on-demand. Default: \$BBHOME/www/snap/ 

 

\item[BBVAR] Directory for all data stored about the monitored items. Default: \$BBSERVERROOT/data/ 

 

\item[BBLOGS] Directory for storing the raw status-logs. Not used unless ``hobbitd\_filestore --status'' is running, which is \textbf{discouraged}
 since it increases the load on the Hobbit server significantly. Default: \$BBVAR/logs/ 

 

\item[BBHTML] Directory for storing HTML status-logs. Not used unless ``hobbitd\_filestore --status --html'' is running, which is \textbf{discouraged}
 since it increases the load on the Hobbit server significantly. Default: \$BBHOME/www/html/ 

 

\item[BBHIST] Directory for storing the history of monitored items. Default: \$BBVAR/hist/ 

 

\item[BBHISTLOGS] Directory for storing the detailed status-log of historical events. Default: \$BBVAR/histlogs/ 

 

\item[BBACKS] Directory for storing information about alerts that have been acknowledged. Default: \$BBVAR/acks/ 

 

\item[BBDISABLED] Directory for storing information about tests that have been disabled. Default: \$BBVAR/disabled/ 

 

\item[BBDATA] Directory for storing incoming ``data'' messages. Default: \$BBVAR/data/ 

 

\item[BBRRDS] Top-level directory for storing RRD files (the databases with trend-information used to generate graphs). Default: \$BBVAR/rrd/ 

 

\item[CLIENTLOGS] Directory for storing the data sent by a Hobbit client around the time a status changes to a warning (yellow) or critical (red) state. Used by the \emph{hobbitd\_hostdata(8)}
 module. Default: \$BBVAR/hostdata/ 

 


 


\end{description}
�
\section{SYSTEM FILES}


 \begin{description}
\item[BBHOSTS] Full path to the Hobbit \emph{bb-hosts(5)}
 configuration file. Default: \$BBHOME/etc/bb-hosts. 

 

\item[BB] Full path to the \emph{bb(1)}
 client program. Default: \$BBHOME/bin/bb. 

 

\item[BBGEN] Full path to the \emph{bbgen(1)}
 webpage generator program. Default: \$BBHOME/bin/bbgen. 

 


 


\end{description}
�
\section{URLS}
\begin{description}
\item[BBSERVERWWWURL] The root URL for the Hobbit webpages, without the hostname. This URL must be mapped to the ~/server/www/ directory in your webserver configuration. See the sample Apache configuration in ~/server/etc/hobbit-apache.conf. 

 

\item[BBSERVERCGIURL] The root URL for the Hobbit CGI-scripts, without the hostname. This directory must be mapped to the ~/cgi-bin/ directory in your webserver configuration, and must be flagged as holding executable scripts. See the sample Apache configuration in ~/server/etc/hobbit-apache.conf. 

 

\item[BBWEBHOST] Initial part of the Hobbit URL, including just the protocol and the hostname, e.g. ``\url{http://www.foo.com}`` 

 

\item[BBWEBHOSTURL] Prefix for all of the static Hobbit webpages, e.g. ``\url{http://www.foo.com/hobbit}`` 

 

\item[BBWEBHTMLLOGS] URL prefix for the static HTML status-logs generated when BBLOGSTATUS=STATIC. Note that this setting is \textbf{discouraged}
 so this setting should not be used. 

 

\item[BBWEB] URL prefix (without hostname) of the Hobbit webpages. E.g. ``/hobbit''. 

 

\item[BBSKIN] URL prefix (without hostname) of the Hobbit graphics. E.g. ``/hobbit/gifs''. 

 

\item[BBHELPSKIN] URL prefix (without hostname) of the Hobbit on-line help files. E.g ``/hobbit/help''. 

 

\item[BBMENUSKIN] URL prefix (without hostname) of the Hobbit menu files. E.g ``/hobbit/menu''. 

 

\item[BBNOTESSKIN] URL prefix (without hostname) of the Hobbit on-line notes files. E.g ``/hobbit/notes''. 

 

\item[BBREPURL] URL prefix (without hostname) of the Hobbit availability reports. E.g. ``/hobbit/rep''. 

 

\item[BBSNAPURL] URL prefix (without hostname) of the Hobbit snapshots. E.g. ``/hobbit/snap''. 

 

\item[BBWAP] URL prefix (without hostname) of the Hobbit WAP/WML files. E.g. ``/hobbit/wml''. 

 

\item[CGIBINURL] URL prefix (without hostname) of the Hobbit CGI-scripts. Default: \$BBSERVERCGIURL . 

 

\item[COLUMNDOCURL] Format string used to build a link to the documentation for a column heading. Default: ``\$CGIBINURL/hobbitcolumn.sh?\%s'', which causes links to use the \emph{hobbitcolumn.sh(1)}
 script to document a column. 

 


 


\end{description}
�\section{SETTINGS FOR SENDING MESSAGES TO HOBBIT}
\begin{description}
\item[BBDISP] The IP-address used to contact the \emph{hobbitd(8)}
 service. Used by clients and the tools that perform network tests. Default: \$BBSERVERIP 

 

\item[BBDISPLAYS] List of IP-adresses. Clients and network test tools will try to send status reports to a Hobbit server running on each of these adresses. This setting is only used if BBDISP=0.0.0.0. 

 

\item[PAGELEVELS] Compatibility setting for Big Brother: List of colors that are considered ``critical'' and therefore will trigger an alert. Not used by Hobbit. 

 

\item[BBPAGE] Compatibility setting for Big Brother: This is the IP-address of the server where a BBPAGER service is running. It is not used by Hobbit. 

 

\item[BBPAGERS] Compatibility setting for Big Brother: List of servers running the BBPAGER service, used if BBPAGE=0.0.0.0. It is not used by Hobbit. 

 

\item[BBPORT] The portnumber for used to contact the \emph{hobbitd(8)}
 service. Used by clients and the tools that perform network tests. Default: 1984. 

 

\item[DOCOMBO] Compatibility setting for Big Brother. Controls whether so send combo-messages or not. Ignored by Hobbit. 

 

\item[BBMAXMSGSPERCOMBO] The maximum number of status messages to combine into one combo message. You may need to lower this number of your BBDISPLAY server has trouble keeping up with the incoming status messages from bbtest-net. Default: 100. 

 

\item[BBSLEEPBETWEENMSGS] Length of a pause introduced between each successive transmission of a combo-message by bbtest-net. You may have to increase this value to give your BBDISPLAY server time to process one combo message before the next one arrives. This number defines how many microseconds to wait between the messages. Default: 0 (send messages as quickly as possible). 

 


 


\end{description}
�\section{HOBBITD SETTINGS}


 \begin{description}
\item[ALERTCOLORS] Comma-separated list of the colors that may trigger an alert-message. The default is ``red,yellow,purple''. Note that alerts may further be generated or suppresed based on the configuration in the \emph{hobbit-alerts.cfg(5)}
 file. 

 

\item[OKCOLORS] Comma-separated list of the colors that may trigger a recovery-message. The default is ``green,clear,blue''. 

 

\item[ALERTREPEAT] How often alerts get repeated while a status is in an alert state. This is the default setting, which may be changed in the \emph{hobbit-alerts.cfg(5)}
 file. 

 

\item[BBGHOSTS] Controls how status messages from unknown hosts (i.e. hosts not listed in the bb-hosts file) are handled. 

 \textbf{BBGHOSTS=1:}
 Causes the status report to be silently discarded. This is the default behaviour in Hobbit. 


 \textbf{BBGHOSTS=2:}
 Discards the status report, but keep track of the hostname and report it on the hobbitd status page. 


  When BBGHOSTS is set to 1 or 2, the hostnames in incoming status-messages is matched without any case-sensitivity, unlike normal Big Brother which is case-sensitive in hostnames. So with BBGHOSTS set to 1 or 2, ``WWW.FOO.COM'' and ``www.foo.com`` are considered to be the same host. If necessary, the incoming hostname will be changed to match the way it is written in the bb-hosts file, changing case as needed. 


 

\item[MAXMSG\_STATUS] The maximum size of a ``status'' message in kB, default: 256. Status messages are the ones that end up as columns on the web display. The default size should be adequate in most cases, but some extension scripts can generate very large status messages - close to 1024 kB. You should only change this if you see messages in the hobbitd log file about status messages being truncated. 

 

\item[MAXMSG\_CLIENT] The maximum size of a ``client'' message in kB, default: 512. ``client'' messages are generated by the Hobbit client, and often include large process-listings. You should only change this if you see messages in the hobbitd log file about client messages being truncated. 

 

\item[MAXMSG\_DATA] The maximum size of a ``data'' message in kB, default: 256. ``data'' messages are typically used for client reports of e.g. netstat or vmstat data. You should only change this setting if you see messages in the hobbitd log file about data messages being truncated. 

 

\item[MAXMSG\_NOTES] The maximum size of a ``notes'' message in kB, default: 256. ``notes'' messages provide a way for uploading documentation about a host to Hobbit; it is not enabled by default. If you want to upload large documents, you may need to change this setting. 

 

\item[MAXMSG\_STACHG] The maximum size of a ``status change'' message in kB, default: Current value of the MAXMSG\_STATUS setting. Status-change messages occur when a status changes color. There is no reason to change this setting. 

 

\item[MAXMSG\_PAGE] The maximum size of a ``page'' message in kB, default: Current value of the MAXMSG\_STATUS setting. ``page'' messages are alerts, and include the status message that triggers the alert. There is no reason to change this setting. 

 

\item[MAXMSG\_ENADIS] The maximum size of an ``enadis'' message in kB, default: 32. ``enadis'' are small messages used when enabling or disabling hosts and tests, so the default size should be adequate. 

 

\item[MAXMSG\_CLICHG] The maximum size of a ``client change'' message in kB, default: Current value of the MAXMSG\_CLIENT setting. Client-change messages occur when a status changes color to one of the alert-colors, usually red, yellow and purple. There is no reason to change this setting. 

 


 


\end{description}
�
\section{HOBBITD\_HISTORY SETTINGS}


 \begin{description}
\item[BBALLHISTLOG] If set to TRUE, \emph{hobbitd\_history(8)}
 will update the \$BBHIST/allevents file logging all changes to a status. The allevents file is used by the \emph{bb-eventlog.cgi(1)}
 tool to show the list of recent events on the BB2 webpage. 

 

\item[BBHOSTHISTLOG] If set to TRUE, \emph{hobbitd\_history(8)}
 will update the host-specific eventlog that keeps record of all status changes for a host. This logfile is not used by any Hobbit tool. 

 

\item[SAVESTATUSLOG] If set to TRUE, \emph{hobbitd\_history(8)}
 will save historical detailed status-logs to the \$BBHISTLOGS directory. 

 


 


\end{description}
�
\section{HOBBITD\_ALERT SETTINGS}


 \begin{description}
\item[MAIL] Command used to send alerts via e-mail, including a ``Subject:'' header in the mail. Default: ``mail -s'' 

 

\item[MAILC] Command used to send alerts via e-mail in a form that does not have a ``Subject'' in the mail. Default: ``mail'' 

 

\item[SVCCODES] Maps status-columns to numeric service-codes. The numeric codes are used when sending an alert using a script, where the numeric code of the service is provided in the BBSVCNUM variable. 

 


 


\end{description}
�
\section{HOBBITD\_RRD SETTINGS}


 \begin{description}
\item[TEST2RRD] List of ``COLUMNNAME[=RRDSERVICE]'' settings, that define which status- and data-messages have a corresponding RRD graph. You will normally not need to modify this, unless you have added a custom TCP-based test to the bb-services file, and want to collect data about the response-time, OR if you are using the \emph{hobbitd\_rrd(8)}
 external script mechanism to collect data from custom tests. Note: All TCP tests are automatically added. 

  This is also used by the \emph{bb-hostsvc.cgi(1) }
 script to determine if the detailed status view of a test should include a graph. 


 

\item[GRAPHS] List of the RRD databases, that should be shown as a graph on the ``trends'' column. 

 

\item[NORRDDISKS] This is used to disable the tracking of certain filesystems. By default all filesystems reported by a client are tracked. In some cases you may want to disable this for certain filesystems, e.g. database filesystems since they are always completely full. This setting is a regular expression that is matched against the filesystem name (the Unix mount-point, or the Windows disk-letter) - if the filesystem name matches this expression, then it will not be tracked by Hobbit.  
 Note: Setting this does not affect filesystems that are already being tracked by Hobbit - to remove them, you must remove the RRD files for the unwanted filesystems from the ~hobbit/data/rrd/HOSTNAME/ directory. 

 

\item[RRDDISKS] This is used to enable tracking of only selected filesystems (see the NORRDDISKS setting above). By default all filesystems are being tracked, setting this changes that default so that only those filesystems that match this pattern will be tracked. 

 


 


\end{description}
�
\section{BBTEST-NET NETWORK TEST SETTINGS}


 \begin{description}
\item[BBLOCATION] If this variable is defined, then only the hosts that have been tagged with ``NET:\$BBLOCATION'' will be tested by the bbtest-net tool. 

 

\item[CONNTEST] If set to TRUE, the connectivity (ping) test will be performed. 

 

\item[IPTEST\_2\_CLEAR\_ON\_FAILED\_CONN] If set to TRUE, then failing network tests go CLEAR if the conn-test fails. 

 

\item[NONETPAGE] List of network services (separated with $<$space$>$) that should go yellow upon failure instead of red. 

 

\item[BBROUTERTEXT] When using the ``router'' or ``depends'' tags for a host, a failure status will include text that an ``Intermediate router is down''. With todays network topologies, the router could be a switch or another network device; if you define this environment variable the word ``router'' will be replaced with whatever you put into the variable. So to inform the users that an intermediate switch or router is down, use BBROUTERTEXT=''switch or router''. This can also be set on a per-host basis using the ``DESCR:hosttype:description'' tag in the \emph{bb-hosts(5)}
 file. 

 

\item[NETFAILTEXT] When a network test fails, the status message reports ``SERVICENAME not OK''. The ``not OK'' message can be changed via this variable, e.g. you can change it to ``FAILED'' or customize it as you like. 

 

\item[FPING] The command used to run the \emph{hobbitping(1)}
 tool for the connectivity test. (The name FPING is due to the fact that the ``fping'' utility was used until Hobbit version 4.2). This may include suid-root wrappers and hobbitping options. Default: ``hobbitping'' 

 

\item[TRACEROUTE] Defines the location of the ``traceroute'' tool and any options needed to run it. traceroute it used by the connectivity test when the ping test fails; if requested via the ``trace'' tag, the TRACEROUTE command is executed to try to determine the point in the network that is causing the problem. By default the command executed is ``traceroute -n -q 2 -w 2 -m 15'' (no DNS lookup, max. 2 probes, wait 2 seconds per hop, max 15 hops). 

  If you have the \emph{mtr(8)}
 tool installed - available from \url{http://www.bitwizard.nl/mtr/} - I strongly recommend using this instead. The recommended setting for mtr is ``/usr/sbin/mtr -c 2 -n --report'' (the exact path to the mtr utility may be different on your system). Note that mtr needs to be installed suid-root on most systems. 


 

\item[NTPDATE] Defines the \emph{ntpdate(1)}
 program used for the ``ntp'' test. Default: ``ntpdate'' 

 

\item[RPCINFO] Defines the \emph{rpcinfo(8)}
 program used for ``rpc'' tests. Default: ``rpcinfo'' 

 


 


\end{description}
�
\section{BBGEN WEBPAGE GENERATOR SETTINGS}


 \begin{description}
\item[HOBBITLOGO] HTML code that is inserted on all standard headers. The default is to add the text ``Hobbit'' in the upper-left corner of the page, but you can easily replace this with e.g. a company logo. If you do, I suggest that you keep it at about 30-35 pixels high, and 100-150 pixels wide. 

 

\item[MKBBLOCAL] The string ``Pages hosted locally'' that appears above all of the pages linked from the main Hobbit webpage. 

 

\item[MKBBSUBLOCAL] The string ``Subpages hosted locally'' that appears above all of the sub-pages linked from pages below the main Hobbit webpage. 

 

\item[MKBBREMOTE] The string ``Remote status display'' that appears about the summary statuses displayed on the min Hobbit webpage. 

 

\item[MKBBTITLE] HTML tags designed to go in a $<$FONT$>$ tag, to choose the font for titles of the webpages. 

 

\item[MKBBROWFONT] HTML tags designed to go in a $<$FONT$>$ tag, to choose the font for row headings (hostnames) on the webpages. 

 

\item[MKBBCOLFONT] HTML tags designed to go in a $<$FONT$>$ tag, to chose the font for column headings (test names) on the webpages. 

 

\item[MKBBACKFONT] HTML tags designed to go in a $<$FONT$>$ tag, to chose the font for the acknowledgement text displayed on the status-log HTML page for an acknowledged status. 

 

\item[ACKUNTILMSG] When displaying the detailed status of an acknowledged test, Hobbit will include the time that the acknowledge expires using the print-format defined in this setting. You can define the timeformat using the controls in your systems \emph{strftime(3)}
 routine, and add the text suitable for your setup. 

 

\item[BBDATEFORMAT] On webpages generated by bbgen, the default header includes the current date and time. Normally this looks like ``Tue Aug 24 21:59:47 2004''. The BBDATEFORMAT controls the format of this timestamp - you can define the format using the controls in the \emph{strftime(3)}
 routine. E.g. to have it show up as ``2004-08-24 21:59:47 +0200'' you would set BBDATEFORMAT=''\%Y-\%m-\%d \%H:\%M:\%S \%z'' 

 

\item[HOLIDAYFORMAT] How holiday dates are displayed. The default is ``\%d/\%m'' which show the day and month. American users may want to change this to ``\%m/\%d'' to suit their preferred date-display style. This is a formatting string for the system \emph{strftime(3)}
 routine, so any controls available for this routine may be used. 

 

\item[MKBB2COLREPEAT] Inspired by Jeff Stoner's col\_repeat\_patch.tgz patch, this defines the maximum number of rows before repeating the column headings on a webpage. This sets the default value for the \emph{bbgen(1)}
 ``--maxrows'' option; if the command-line option is also specifed, then it overrides this environment variable. Note that unlike Jeff's patch, bbgen implements this for both the bb2.html page and all other pages (bb.html, subpages, bbnk.html). 

 

\item[SUMMARY\_SET\_BKG] If set to TRUE, then summaries will affect the color of the main Hobbit webpage. Default: FALSE. 

 

\item[DOTHEIGHT] The height (in pixels) of the icons showing the color of a status. Default: 16, which matches the default icons. 

 

\item[DOTWIDTH] The width (in pixels) of the icons showing the color of a status. Default: 16, which matches the default icons. 

 

\item[CLIENTSVCS] List of the status logs fed by data from the Hobbit client. These status logs will - if there are Hobbit client data available for the host - include a link to the raw data sent by the client. Default: cpu,disk,memory,procs,svcs. 

 

\item[BBRSSTITLE] If defined, this is the title of the RSS/RDF documents generated when \emph{bbgen(1)}
 is invoked with the ``--rss'' option. The default value is ``Hobbit Alerts''. 

 

\item[WMLMAXCHARS] Maximum size of a WAP/WML output ``card'' when generating these. Default: 1500. 

 

\item[BBMKBB2EXT] List of scripts to run as extensions to the BB2 page. Note that two scripts, ``eventlog.sh'' and ``acklog.sh'' are handled specially: They are handled internally by bbgen, but the script names must be listed in this variable for this function to be enabled. 

 

\item[BBHISTEXT] List of scripts to run as extensions to a history page. 

 

\item[BBREPWARN] Default threshold for listing the availability as ``critical'' (red) when generating the availability report. This can be set on a per-host basis with the WARNPCT setting in \emph{bb-hosts(5).}
 Default: 97 (percent) 

 

\item[BBGENREPOPTS] Default bbgen options used for reports. This will typically include such options as ``--subpagecolumns'', and also ``--ignorecolumns'' if you wish to exclude certain tests from reports by default. 

 

\item[BBGENSNAPOPTS] Default bbgen options used by snapshots. This should be identical to the options you normally used when building Hobbit webpages. 

 


\end{description}
�
\section{FILES}
\textbf{~hobbit/server/etc/hobbitserver.cfg}


 �
\section{SEE ALSO}
hobbit(7) 

 


  
�
\section{Index}
\begin{description}
\item[NAME]
\item[DESCRIPTION]
\item[ENVIRONMENT AREAS]
\item[GENERAL SETTINGS]
\item[DIRECTORIES]
\item[SYSTEM FILES]
\item[URLS]
\item[SETTINGS FOR SENDING MESSAGES TO HOBBIT]
\item[HOBBITD SETTINGS]
\item[HOBBITD\_HISTORY SETTINGS]
\item[HOBBITD\_ALERT SETTINGS]
\item[HOBBITD\_RRD SETTINGS]
\item[BBTEST-NET NETWORK TEST SETTINGS]
\item[BBGEN WEBPAGE GENERATOR SETTINGS]
\item[FILES]
\item[SEE ALSO]

\end{description}
 
 This document was created by man2html, using the manual pages. 
 Time: 16:21:46 GMT, January 11, 2008 

