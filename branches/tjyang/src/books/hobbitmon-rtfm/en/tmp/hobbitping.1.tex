



\chapter{HOBBITPING}

\section{HOBBITPING}
 Section: User Commands (1) 
Updated: Version Exp: 11 Jan 2008 
Index Return to Main Contents 
�\section{NAME}
 hobbitping - Hobbit ping tool �\section{SYNOPSIS}
\textbf{hobbitping [--retries=N] [--timeout=N] [IP-adresses]}


 �
\section{DESCRIPTION}
\emph{hobbitping(1)}
 is used for ping testing of the hosts monitored by the \emph{hobbit(7)}
 monitoring system. It reads a list of IP adresses from stdin, and performs a ``ping'' check to see if these hosts are alive. It is normally invoked by the \emph{bbtest-net(1)}
 utility, which performs all of the Hobbit network tests. 

  Optionally, if a list of IP-adresses is passed as commandline arguments, it will ping those IP's instead of reading them from stdin. 


  hobbitping only handles IP-adresses, not hostnames. 


  hobbitping was inspired by the \emph{fping(1)}
 tool, but has been written from scratch to implement a fast ping tester without much of the overhead found in other such utilities. The output from hobbitping is similar to that of ``fping -Ae''. 


  hobbitping probes multiple systems in parallel, and the runtime is therefore mostly dependant on the timeout-setting and the number of retries. With the default options, hobbitping takes approximately 18 seconds to ping all hosts (tested with an input set of 1500 IP adresses). 


 �
\section{SUID-ROOT INSTALLATION REQUIRED}
 hobbitping needs to be installed with suid-root privileges, since it requires a ``raw socket'' to send and receive ICMP Echo (ping) packets. 

  hobbitping is implemented such that it immediately drops the root privileges, and only regains them to perform two operations: Obtaining the raw socket, and optionally binding it to a specific source address. These operations are performed as root, the rest of the time hobbitping runs with normal user privileges. Specifically, no user-supplied data or network data is used while running with root privileges. Therefore it should be safe to provide hobbitping with the necessary suid-root privileges. 


 �
\section{OPTIONS}
\begin{description}
\item[--retries=N] Sets the number of retries for hosts that fail to respond to the initial ping, i.e. the number of ping probes sent in addition to the initial probe. The default is --retries=2, to ping a host 3 times before concluding that it is not responding. 

 

\item[--timeout=N] Determines the timeout (in seconds) for ping probes. If a host does not respond within N seconds, it is regarded as unavailable, unless it responds to one of the retries. The default is --timeout=5. 

 

\item[--responses=N] hobbitping normally stops pinging a host after receiving a single response, and uses that to determine the round-trip time. If the first response takes longer to arrive - e.g. because of additional network overhead when first determining the route to the target host - it may skew the round-trip-time reports. You can then use this option to require N responses, and hobbitping will calculate the round-trip time as the average of all of responsetimes. 

 

\item[--max-pps=N] Maximum number of packets per second. This limits the number of ICMP packets hobbitping will send per second, by enforcing a brief delay after each packet is sent. The default setting is to send a maximum of 50 packets per second. Note that increasing this may cause flooding of the network, and since ICMP packets can be discarded by routers and other network equipment, this can cause erratic behaviour with hosts recorded as not responding when they are in fact OK. 

 

\item[--source=ADDRESS] Use ADDRESS as the source IP address of the ping packets sent. On multi-homed systems, allows you to select the source IP of the hosts going out, which might be necessary for ping to work. 

 

\item[--debug] Enable debug output. This prints out all packets sent and received. 

 


\end{description}
�\section{SEE ALSO}
hobbit(7), bbtest-net(1), fping(1) 

 


  
�
\section{Index}
\begin{description}
\item[NAME]
\item[SYNOPSIS]
\item[DESCRIPTION]
\item[SUID-ROOT INSTALLATION REQUIRED]
\item[OPTIONS]
\item[SEE ALSO]

\end{description}
 
 This document was created by man2html, using the manual pages. 
 Time: 16:21:46 GMT, January 11, 2008 

