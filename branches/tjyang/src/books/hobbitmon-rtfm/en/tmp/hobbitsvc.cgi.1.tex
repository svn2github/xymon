\chapter{BB-HOSTSVC.CGI}

\section{BB-HOSTSVC.CGI}
 Section: User Commands (1) 
Updated: Version Exp: 11 Jan 2008 
Index Return to Main Contents 
�\section{NAME}
hgcmd{hobbitsvc.cgi} - CGI program to view Hobbit status logs �\section{SYNOPSIS}
\textbf{hobbitsvc.cgi [--hobbitd|--historical] [--history=\\{top|bottom\\}]}


 �
\section{DESCRIPTION}
\textbf{hobbitsvc.cgi}
 is a CGI program to present a Hobbit status log in HTML form (ie, as a web page). It can be used both for the logs showing the current status, and for historical logs from the ``histlogs'' directory. It is normally invoked as a CGI program, and therefore receives most of the input parameters via the CGI QUERY\_STRING environment variable. 

  Unless the ``--historical'' option is present, the current status log is used. This assumes a QUERY\_STRING environment variable of the form  
 
���HOSTSVC=hostname.servicename  
 where ``hostname'' is the name of the host with commas instead of dots, and ``servicename'' is the name of the service (the column name in Hobbit). Such links are automatically generated by the \emph{bbgen(1)}
 tool when the environment contains ``BBLOGSTATUS=dynamic''. 


  With the ``--historical'' option present, a historical logfile is used. This assumes a QUERY\_STRING environment variable of the form  
 
���HOST=hostname\&SERVICE=servicename\&TIMEBUF=timestamp  
 where ``hostname'' is the name of the host with commas instead of dots, ``servicename'' is the name of the service, and ``timestamp'' is the time of the log. This is automatically generated by the \emph{bb-hist.cgi(1)}
 tool. 


 �
\section{OPTIONS}
\begin{description}
\item[--hobbitd] Retrieve the current status log from \emph{hobbitd(1)}
 rather than from the logfile. This is for use with the Hobbit daemon from the Hobbit monitor version 4. 

 

\item[--historical] Use a historical logfile instead of the current logfile. 

 

\item[--history=\\{top|bottom|none\\}] When showing the current logfile, provide a ``HISTORY'' button at the top or the bottom of the webpage, or not at all. The default is to put the HISTORY button at the bottom of the page. 

 

\item[--env=FILENAME] Load the environment from FILENAME before executing the CGI. 

 

\item[--templates=DIRECTORY] Where to look for the HTML header- and footer-templates used when generating the webpages. Default: \$BBHOME/web/ 

 

\item[--no-svcid] Do not include the HTML tags to identify the hostname/service on the generated web page. Useful is this already happens in the hostsvc\_header template file, for instance. 

 

\item[--multigraphs=TEST1[,TEST2]] This causes hobbitsvc.cgi to generate links to service graphs that are split up into multiple images, with at most 5 graphs per image. This option only works in Hobbit mode. If not specified, only the ``disk'' status is split up this way. 

 

\item[--no-disable] By default, the info-column page includes a form allowing users to disable and re-enable tests. If your setup uses the default separation of administration tools into a separate, password- protected area, then use of the disable- and enable-functions requires access to the administration tools. If you prefer to do this only via the dedicated administration page, this option will remove the disable-function from the info page. 

 

\item[--no-jsvalidation] The disable-function on the info-column page by default uses JavaScript to validate the form before submitting the input to the Hobbit server. However, some browsers cannot handle the Javascript code correctly so the form does not work. This option disables the use of Javascript for form-validation, allowing these browsers to use the disable-function. 

 


\end{description}
�\section{FILES}
\begin{description}
\item[\$BBHOME/web/hostsvc\_header] HTML template header 

 

\item[\$BBHOME/web/hostsvc\_footer] HTML template footer 

 


\end{description}
�\section{ENVIRONMENT}
\begin{description}
\item[NONHISTS=info,trends,graphs] A comma-separated list of services that does not have meaningful history, e.g. the ``info'' and ``trends'' columns. Services listed here do not get a ``History'' button. 

 

\item[TEST2RRD=test,test] A comma-separated list of the tests that have an RRD graph. 

 


\end{description}
�\section{SEE ALSO}
hobbit(7), hobbitd(1) 

 


  
�
\section{Index}
\begin{description}
\item[NAME]
\item[SYNOPSIS]
\item[DESCRIPTION]
\item[OPTIONS]
\item[FILES]
\item[ENVIRONMENT]
\item[SEE ALSO]

\end{description}
 
 This document was created by man2html, using the manual pages. 
 Time: 16:21:47 GMT, January 11, 2008 

