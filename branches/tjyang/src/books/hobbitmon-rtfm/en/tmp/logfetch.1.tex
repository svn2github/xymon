



\chapter{LOGFETCH}

\section{LOGFETCH}
 Section: User Commands (1) 
Updated: Version Exp: 11 Jan 2008 
Index Return to Main Contents 
�\section{NAME}
 \hgcmd{logfetch} - Hobbit client data collector �\section{SYNOPSIS}
\textbf{logfetch CONFIGFILE STATUSFILE}


 �
\section{DESCRIPTION}
\textbf{logfetch}
 is part of the Hobbit client. It is responsible for collecting data from logfiles, and other file-related data, which is then sent to the Hobbit server for analysis. 

  logfetch uses a configuration file, which is automatically retrieved from the Hobbit server. There is no configuration done locally. The configuration file is usually stored in the \textbf{\$BBHOME/tmp/logfetch.cfg}
 file, but editing this file has no effect since it is re-written with data from the Hobbit server each time the client runs. 


  logfetch stores information about what parts of the monitored logfiles have been processed already in the \textbf{\$BBHOME/tmp/logfetch.status}
 file. This file is an internal file used by logfetch, and should not be edited. If deleted, it will be re-created automatically. 


 �
\section{SECURITY}
 logfetch needs read access to the logfiles it should monitor. If you configure monitoring of files or directories through the ``file:'' and ``dir:'' entries in \emph{client-local.cfg(5)}
 then logfetch will require at least read-acces to the directory where the file is located. If you request checksum calculation for a file, then it must be readable by the hobbit client user. 

  Do \textbf{NOT}
 install logfetch as suid-root. There is no way that logfetch can check whether the configuration file it uses has been tampered with, so installing logfetch with suid-root privileges could allow an attacker to read any file on the system by using a hand-crafted configuration file. In fact, logfetch will attempt to remove its own suid-root setup if it detects that it has been installed suid-root. 


 �
\section{ENVIRONMENT VARIABLES}
\begin{description}
\item[DU] Command used to collect information about the size of directories. By default, this is the command \textbf{du -k}
. If the local du-command on the client does not recognize the ``-k'' option, you should set the DU environment variable in the \textbf{\$BBHOME/etc/hobbitclient.cfg}
 file to a command that does report directory sizes in kilobytes. 

 


\end{description}
�\section{FILES}
\begin{description}
\item[\$BBHOME/tmp/logfetch.cfg]
\item[\$BBHOME/tmp/logfetch.status]

 


\end{description}
�\section{SEE ALSO}
hobbit(7), hobbit-clients.cfg(5) 

 


  
�
\section{Index}
\begin{description}
\item[NAME]
\item[SYNOPSIS]
\item[DESCRIPTION]
\item[SECURITY]
\item[ENVIRONMENT VARIABLES]
\item[FILES]
\item[SEE ALSO]

\end{description}
 
 This document was created by man2html, using the manual pages. 
 Time: 16:21:46 GMT, January 11, 2008 

