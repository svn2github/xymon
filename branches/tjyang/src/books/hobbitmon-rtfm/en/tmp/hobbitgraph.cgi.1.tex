



\chapter{HOBBITGRAPH.CGI}

\section{HOBBITGRAPH.CGI}
 Section: User Commands (1) 
Updated: Version Exp: 11 Jan 2008 
Index Return to Main Contents 
�\section{NAME}
 hobbitgraph.cgi - CGI to generate Hobbit trend graphs �\section{SYNOPSIS}
\textbf{hobbitgraph [options]}


 �
\section{DESCRIPTION}
\textbf{hobbitgraph.cgi}
 is invoked as a CGI script via the hobbitgraph.sh CGI wrapper. 

  hobbitgraph.cgi is passed a QUERY\_STRING environment variable with the following parameters: 


 \textbf{host}
 Name of the host to generate a graph for 


 \textbf{service}
 Name of the service to generate a graph for 


 \textbf{disp}
 Display-name of the host, used on the generated graphs instead of hostname. 


 \textbf{graph}
 Can be ``hourly'', ``daily'', ``weekly'' or ``monthly'' to select the time period that the graph covers. 


 \textbf{first}
 Used to split multi-graphs into multiple graphs. This causes hobbitgraph.cgi to generate only the graphs starting with the ``first'th'' graph and continuing for ``count''. 


 \textbf{count}
 Number of graphs in a multi-graph. 


 \textbf{upper}
 Set the upper limit of the graph. See \emph{rrdgraph(1)}
 for a description of the ``-u'' option. 


 \textbf{lower}
 Set the lower limit of the graph. See \emph{rrdgraph(1)}
 for a description of the ``-l'' option. 


 \textbf{graph\_start}
 Set the starttime of the graph. This is used in zoom-mode. 


 \textbf{graph\_end}
 Set the end-time of the graph. This is used in zoom-mode. 


 \textbf{action=menu}
 Generate an HTML page with links to 4 graphs, representing the hourly, weekly, monthly and yearly graphs. Doesn't actually generate any graphs, only the HTML that links to the graphs. 


 \textbf{action=selzoom}
 Generate an HTML page with link to single graph, and with JavaScript code that lets the user select part of the graph for a zoom-operation. Doesn't actually generate graph, only the HTML that links to the graph. 


 \textbf{action=view}
 Generate a single graph image. 


 �
\section{OPTIONS}
\begin{description}
\item[--config=FILENAME] Loads the graph configuration file from FILENAME. If not specified, the file \$BBHOME/etc/hobbitgraph.cfg is used. See the \emph{hobbitgraph.cfg(5)}
 for details about this file. 

 

\item[--env=FILENAME] Loads the environment settings defined in FILENAME before executing the CGI. 

 

\item[--rrddir=DIRECTORY] The top-level directory for the RRD files. If not specified, the directory given by the BBRRDS environment is used. 

 

\item[--save=FILENAME] Instead of returning the image via the CGI interface (i.e. on stdout), save the generated image to FILENAME. 

 

\item[--debug] Enable debugging output. 

 


\end{description}
�\section{ENVIRONMENT}


 \textbf{QUERY\_STRING}
 Provided by the webserver CGI interface, this decides what graph to generate. 


 �
\section{FILES}


 \textbf{hobbitgraph.cfg:}
 The configuration file determining how graphs are generated from RRD files. 


 �
\section{SEE ALSO}
hobbitgraph.cfg(5), hobbit(7), rrdtool(1) 

 


  
�
\section{Index}
\begin{description}
\item[NAME]
\item[SYNOPSIS]
\item[DESCRIPTION]
\item[OPTIONS]
\item[ENVIRONMENT]
\item[FILES]
\item[SEE ALSO]

\end{description}
 
 This document was created by man2html, using the manual pages. 
 Time: 16:21:47 GMT, January 11, 2008 

